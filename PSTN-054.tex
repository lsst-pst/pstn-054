\documentclass[PST,authoryear,toc]{lsstdoc}
% lsstdoc documentation: https://lsst-texmf.lsst.io/lsstdoc.html
\input{meta}

% Package imports go here.

% Local commands go here.

%If you want glossaries
%\input{aglossary.tex}
%\makeglossaries

\title{Updated estimates of the Rubin system throughput and expected LSST image depth}

% Optional subtitle
% \setDocSubtitle{A subtitle}

\author{%
federica bianco (she/her/hers)
}

\setDocRef{PSTN-054}
\setDocUpstreamLocation{\url{https://github.com/lsst-pst/pstn-054}}

\date{\vcsDate}

% Optional: name of the document's curator
% \setDocCurator{The Curator of this Document}

\setDocAbstract{%
This document presents updated estimates of the Rubin system throughput and compares them to through-
put requirements from the LSST Science Requirements Document. In addition, it uses these estimates to
forecast LSST’s median single-visit and co-added image depths for the current baseline LSST cadence simulation. Estimated system performance relies on actual measurements of the performance of various system
hardware components, and on simulations where measurements are still unavailable. The updated performance estimates meet all the relevant requirements from the LSST Science Requirements Document
}

% Change history defined here.
% Order: oldest first.
% Fields: VERSION, DATE, DESCRIPTION, OWNER NAME.
% See LPM-51 for version number policy.
\setDocChangeRecord{%
  \addtohist{1}{YYYY-MM-DD}{Unreleased.}{federica bianco (she/her/hers)}
}


\begin{document}

% Create the title page.
\maketitle
% Frequently for a technote we do not want a title page  uncomment this to remove the title page and changelog.
% use \mkshorttitle to remove the extra pages

% ADD CONTENT HERE
% You can also use the \input command to include several content files.

\appendix
% Include all the relevant bib files.
% https://lsst-texmf.lsst.io/lsstdoc.html#bibliographies
\section{References} \label{sec:bib}
\renewcommand{\refname}{} % Suppress default Bibliography section
\bibliography{local,lsst,lsst-dm,refs_ads,refs,books}

% Make sure lsst-texmf/bin/generateAcronyms.py is in your path
\section{Acronyms} \label{sec:acronyms}
\addtocounter{table}{-1}
\begin{longtable}{p{0.145\textwidth}p{0.8\textwidth}}\hline
\textbf{Acronym} & \textbf{Description}  \\\hline

B & Byte (8 bit) \\\hline
CCD & Charge-Coupled Device \\\hline
DIMM & Differential Image Motion Monitor \\\hline
DOE & Department of Energy \\\hline
FWHM & Full Width at Half-Maximum \\\hline
HTML & HyperText Markup Language \\\hline
LPM & LSST Project Management (Document Handle) \\\hline
LSE & LSST Systems Engineering (Document Handle) \\\hline
LSST & Legacy Survey of Space and Time (formerly Large Synoptic Survey Telescope) \\\hline
M1 & primary mirror \\\hline
M2 & Secondary Mirror \\\hline
M3 & tertiary mirror \\\hline
PSF & Point Spread Function \\\hline
PST & Project Science Team \\\hline
PSTN & Project Science Technical Note \\\hline
QE & quantum efficiency \\\hline
RTN & Rubin Technical Note \\\hline
SCOC & Survey Cadence Optimization Committee \\\hline
SED & Spectral Energy Distribution \\\hline
SNR & Signal to Noise Ratio \\\hline
SRD & LSST Science Requirements; LPM-17 \\\hline
WFD & Wide Fast Deep \\\hline
arcsec & arcsecond second of arc (unit of angle) \\\hline
\end{longtable}

% If you want glossary uncomment below -- comment out the two lines above
%\printglossaries





\end{document}
